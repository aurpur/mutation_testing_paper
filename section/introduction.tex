\label{sec:introduction}

La qualité des applications web est une préoccupation majeure dans cycle de
vie d'un logiciel. Elle est souvent coûteuse et difficile à garantir. Elle
est par ailleurs cruciale pour les industries sensible comme la santé ou la
banque. Pour tester un logiciel on utilise entre des tests End-to-End (E2E). Ces tests sont généralement réalisés en utilisant des
frameworks tels que Selenium ou Cypress \cite{cerioli20205}. Les tests End-to-End (E2E) sont une approche de test qui vise à tester
le comportement d'une application web dans son ensemble selon la perspective de
l'utilisateur final. De ce fait, le programme sous test (PUT) est une boite
noire du point de vue du test. Cette caractéristique rends entre autres difficile
l'évaluation de la qualité des tests E2E tant pendant la création que la
maintenance du test. Ricca \etal \cite{ricca2019three} ont identifié plusieurs
défis liés à la qualité des tests E2E parmis lesquels on retrouve le problème de
la fragilité. Pour évaluer la fragilité des tests en général on utilise les tests de mutation \cite{hamimoune2016mutation}. Les tests de mutation est une
technique de test qui consiste à introduire des défauts dans le code source du
PUT pour évaluer la qualité des tests \cite{woodward1993mutation}. Cette méthode est utilisé dans
l'industrie pour les tests unitaires du fait de leur caractéristique de test en
boite blanche. Cependant quelques travaux commencent à s'intéresser à
l'application des tests de mutation pour les tests E2E. Yandrapally \etal
\cite{yandrapally2021mutation} propose dans le framework MaewU qui évalue des
suites de tests d'interface utilisateur (UI). Il introduit 16 opérations de mutation basées
sur 250 bug reports. Leotta \etal \cite{leotta2024mutta} propose par la suite l'outil
MUTTA qui permet d'automatiser le processus de test de mutation des tests E2E.
Ces travaux ont montré que les tests de mutation sont certes l'approche la plus
pertinante pour évaluer la qualité des tests E2E, mais il subsiste plusieurs défis
à relever. Parmi ces défis on retourve : (i) le coût élevé des tests de
mutation, (ii) l'identification et le filtre des mutants équivalents, et (iii)
l'éfficacité des tests de mutation. Dans ce travail, nous appliquons la génération des tests de mutation pour répondre à la problèmatique suivante : \textit{Comment optimiser la génération des tests de mutation pour les tests End-to-End (E2E) en surmontant les défis du coût élevé, de l’identification des mutants équivalents, et en maximisant l’efficacité des tests pour une évaluation fiable de la qualité des suites de tests ?}

Nous proposons une approche basée sur les modèles
de langage pour générer des tests de mutation plus réalistes, nécessaire et
diversifiés. Notre étude est conçue pour répondre aux questions de
recherche suivantes:

\begin{itemize}
    \item \textbf{RQ1:} LLM peut-il générer des tests de mutation plus efficaces que l'approche de pointe, en s'appuyant sur l'historique des erreurs

    \item \textbf{RQ2:} Quel est le modele de langage le plus adapté pour la génération des tests de mutation?

    \item \textbf{RQ3:} Peut-on identifier et réduire le nombre de mutants équivalents générés par notre approche?

    \item \textbf{RQ4:} Le coût (temps d'exécution et nombre de mutants) des tests de mutation générés par notre approche est-il réduit par rapport à l'approche de l'état de l'art?

\end{itemize}

