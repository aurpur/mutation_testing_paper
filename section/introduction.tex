\label{sec:introduction}

The quality of web applications is a major concern in the software software life
cycle. It is often costly and difficult to guarantee. It crucial for sensitive
industries such as healthcare and banking. banking. To test a web application,
we use several types of tests, which can be grouped into three levels of a
pyramid. From the bottom of the pyramid pyramid, unit tests, service tests and
End-to-End (E2E) tests. E2E tests are generally produced using frameworks such as Selenium or Cypress.
\cite{cerioli20205}. End-to-End testing is a testing approach that aims to
test the behavior of a web application as a whole from the end-user's
perspective. As a result, the program under test (PUT) is a black box from the
testing point of view. Among other things, this characteristic makes it
difficult to assess the quality of E2E tests during both test creation and
maintenance. Ricca \etal \cite{ricca2019three} have identified several challenges to
the quality of E2E tests, including the problem of fragility. In general,
mutation tests are used to assess test fragility \cite{hamimoune2016mutation}. Mutation testing is a testing technique that consists of introducing defects
into the LUP source code and then evaluating the ability of the tests to capture
these mutations \cite{woodward1993mutation}. This method is used in industry for unit testing, due to its white-box testing
characteristic. However, a number of studies are beginning to focus on the
application of mutation testing to E2E testing. Yandrapally \etal \cite{yandrapally2021mutation} proposes the MaewU framework, which evaluates
user interface (UI) test suites. It introduces 16 mutation operations based on
250 bug reports.


Leotta \etal \cite{leotta2024mutta} subsequently proposed the MUTTA tool, which automates the mutation testing process for E2E tests. This work has shown that mutation testing is indeed the most relevant approach for assessing the quality of E2E tests, but several challenges remain. These include: (i) the high cost (execution time and number of mutants) of mutation tests, (ii) the identification and filtering of equivalent mutants, and (iii) the efficiency of mutation tests. In this work, we apply mutation test generation to address the following problem:
\textit{How do you generate efficient mutation tests for end-to-end testing by meeting the challenges of cost, identification and reduction of equivalent mutants?}

We propose an approach based on language models and PUT bug logs to generate closer-to-existing, necessary and diversified mutation tests. Our study is designed to answer the following research questions:

\begin{itemize}


\item \textbf{RQ1:} What is the most adapted model for generating mutation tests? \Aurel{mutants introduits / mutants générés ?}

\item \textbf{RQ2:} Can LLM be used to identify equivalent mutants? If so, which model is best adapted to this task? \Aurel{mutants équivalents / mutants générés ?}

    \item \textbf{RQ3:} Is the cost (execution time and number of mutants) of the mutation tests generated by our approach reduced compared with the state-of-the-art approach?
    
\item \textbf{RQ4:} Can LLM generate more efficient mutation tests than the
state-of-the-art approach? \Aurel{voir le rapport entre la répartition des
mutants
équivalents par rapport au nombre de mutants générés dans les 2 approches.
Efficacité = mutants tués / mutants générés, temps d'exécution, nombre de
mutants équivalents / mutants générés.}

\end{itemize}

