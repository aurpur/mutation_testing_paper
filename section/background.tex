\label{sec:background}

\subsection{Mutant Generation}
A \textbf{mutant} is a small change in the source code of a PUT. The change is made to simulate a
real-life defect in the code \cite{offutt2001mutation, jia2010analysis}. The aim of mutation testing is to verify that a given test is capable of detecting the mutant by failing. This is known as killing the mutant. Mutants can be divided into two broad categories:

\begin{itemize}
    \item \textbf{Equivalent Mutants:} Mutants that do not affect the behavior of
          the PUT. These mutants are not useful for testing.
    \item \textbf{Non-Equivalent Mutants:} Mutants that affect the behavior of the PUT. These mutants are useful for testing.
\end{itemize}

Mutant generation is a crucial step in the mutation testing process. It is important to note that mutant generation can be either manual or automatic. Manual approaches obviously require human intervention. Whereas automatic approaches use tools such as PIT \cite{leotta2024mutta, coles2016pit}, Major \cite{just2014major},
Jumble \cite{irvine2007jumble} and Javalanche \cite{schuler2009javalanche}.
On the other hand, its tools can generate equivalent mutants, which can lead to incorrect results when evaluating mutation tests and therefore low efficiency. In a recent work, Leotta \etal \cite{leotta2024mutta} used PIT to generate mutants, as the tool provides the high number of mutators needed for E2E mutation testing.


\Aurel{ML in Mutation Testing}

\Aurel{LLM in Mutation Testing}













