\label{sec:background}

\subsection{Mutant Generation}
A \textbf{mutant} is a small change in the source code of a PUT. The change is made to simulate a
real-life defect in the code \cite{offutt2001mutation, jia2010analysis}. The aim of mutation testing is to verify that a given test is capable of detecting the mutant by failing. This is known as killing the mutant. Mutants can be divided into two broad categories:

\begin{itemize}
    \item \textbf{Equivalent Mutants:} Mutants that do not affect the behavior of
          the PUT. These mutants are not useful for testing.
    \item \textbf{Non-Equivalent Mutants:} Mutants that affect the behavior of the PUT. These mutants are useful for testing.
\end{itemize}

La génération de mutants est une étape cruciale dans le processus de test de
mutation. Il est important de noté que la génération des mutants
peut être manuelle ou automatique. Les approches manuelles nécessitent évidement
une intervention humaine. Alors que les approches automatiques utilisent des
outils tels que PIT \cite{leotta2024mutta, coles2016pit}, Major \cite{just2014major},
Jumble \cite{irvine2007jumble}, et Javalanche \cite{schuler2009javalanche}.
En revanche, ses outils peuvent générer des mutants équivalents, ce qui peut entraîner
des résultats incorrects lors de l'évaluation des tests de mutation et donc une
efficacité basse. Dans un recent travail Leotta \etal \cite{leotta2024mutta} a
utilisé PIT pour générer des mutants car l'outils met à disposition un nombre
élevé des mutateurs nécessaire aux tests de mutation E2E.


\subsection{ML in Mutation Testing}

\subsection{LLM in Mutation Testing}













